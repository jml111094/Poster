\textbf{\Large Diskussion}\\

Abb. 6 zeigt, dass die Ergebnisse für die beiden VT1-Methoden häufig sehr stark streuen und für die VT2 wesentlich einheitlicher waren. In Abb. 7 zu erkennen, dass die EQCO\textsubscript{2}-Methode die insgesamt geringsten Differenzen zwischen den Ratern und der Software aufweist und dementsprechend am eindeutigsten ist. Der Korrelationskoeffizient für diese Methode beträgt $r = 0,912$. Zudem ist diese Methode genauer und zuverlässiger als die veraltete Methode und wurde darum in die Software implementiert.\\
Über die Hälfte der erhobenen Messwerte sind vergleichbar mit den Referenzdaten der HUNT 3 Studie~\cite{Loe.2014}, obwohl diese auf Laufbandergometern durchgeführt wurden. Den Erkenntnissen des Projektes folgend, wird angenommen, dass das Gerät des Unternehmens für die Spiroergometrie generell genutzt werden kann. Außerdem können mit der durch die EQCO\textsubscript{2}-Methode erhobenen VT2 nach einem Modell von Wilfried Kindermann Trainingszonen gemäß der Ziele des Unternehmens definiert werden~\cite{Kindermann.2004}.\\
Künftige Verbesserungen am Algorithmus, z.B. im Hinblick auf ein besseres Mittelungs-Verfahren für die Mittelwerte einer Belastungsstufe, könnten die Ergebnisse noch weiter optimieren.