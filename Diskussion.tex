\textbf{\Large Diskussion}\\

Abb. 3 und 4 visualisieren, dass die Schwellen mit den beiden VT1-Methoden häufig sehr stark streuen und die Software diese im Schnitt früher bestimmte als die Rater, die aufgrund der häufigen Artefakte Schwierigkeiten bei der Auswertung hatten. Die EQCO\textsubscript{2}-Methode für die VT2 weist insgesamt die geringsten Differenzen auf und der Korrelationskoeffizient für diese Methode beträgt $r = 0,912$. Zudem ist diese Methode genauer und zuverlässiger als die ursprüngliche Methode, mit der 9 von 28 Grafiken gar nicht auswertbar waren.\\
Über die Hälfte der erhobenen Messwerte sind vergleichbar mit den geschlechts- und altersspezifischen Referenzdaten der HUNT 3 Studie~\cite{Loe.2014}. Von den restlichen Personen lagen, bis auf einer, leicht unter dem Durchschnitt, was aufgrund des höheren Muskelwirkungsgrades auf dem Laufband logisch ist. Den Erkenntnissen des Projektes folgend, wird angenommen, dass das Gerät des Unternehmens für die Spiroergometrie generell genutzt werden kann. Außerdem können mit der durch die EQCO\textsubscript{2}-Methode erhobenen VT2 nach einem Modell von Wilfried Kindermann Trainingszonen gemäß der Ziele des Unternehmens definiert werden~\cite{Kindermann.2004}. Die Methode wurde in Form eines speziellen Algorithmus in die Software implementiert.