\textbf{\Large Einleitung}\\

Die Spiroergometrie (aus lat. \textsl{spirare}: atmen, griech. \textsl{ergo}: Arbeit) ist eine sportmedizinische, technische Anwendung, bei der respiratorische Daten während inkrementierter körperlicher Belastung erfasst und anschließend analysiert werden. Anhand dieser Daten können z.B. individuelle Trainingsbereiche für eine Person bestimmt werden. Die cardioscan GmbH, ein Hamburger Medizintechnik-Hersteller, hat 2017 ein Spiroergometer entwickelt, welches in Verbindung mit einer Software für dieses Verfahren genutzt werden soll. Die Software verwendet momentan einen Algorithmus, der recht sensitiv für Fehler ist und deshalb durch einen neuen ersetzt werden muss. Hierfür eignen sich Methoden zur Bestimmung der ventilatorischen Schwellen VT1 und VT2, die von der AG Spiroergometrie wissenschaftlich empfohlen werden~\cite{Westhoff.2012}.\\
Es wurden im Rahmen dieser Arbeit insgesamt vier Methoden ausgewählt, die in Folge an eine Spiroergometrie mit dem neuen Gerät des Unternehmens zur Schwellenbestimmung verwendet werden. Zu überprüfen war, ob das Gerät generell für diese Anwendung nutzbar ist, welche der vier Methoden zum Erreichen der Firmenziele optimal ist und ob die Methoden genauere Ergebnisse liefern können, als der ursprüngliche Algorithmus.\\





