\begin{center} \textbf{\Large Einleitung} \end{center}
Die Spiroergometrie (aus lat. \textsl{spirare}: atmen, griech. \textsl{ergo}: Arbeit) ist eine oft genutzte technische Anwendung der Sportmedizin zur Bestimmung von individuellen Trainingsbereichen, bei der respiratorische Daten während inkrementierter körperlicher Belastung erfasst und anschließend analysiert werden. Der Hamburger Medizintechnik-Hersteller cardioscan GmbH hat 2017 ein Spiroergometer entwickelt, welches in Verbindung mit einer Software für dieses Verfahren noch nicht getestet wurde. Die Software verwendet momentan einen Algorithmus, der recht sensitiv für Fehler ist und deshalb durch einen neuen ersetzt werden muss. Hierfür eignen sich Methoden zur Bestimmung der Ventilatorischen Schwellen VT1 und VT2, die von der AG Spiroergometrie empfohlen werden~\cite{Westhoff.2012}.\\
Es wurden im Rahmen dieser Arbeit insgesamt vier Methoden ausgewählt, die in Folge an eine Spiroergometrie mit dem neuen Gerät des Unternehmens bestimmt werden sollten. Zu überprüfen war, ob das Gerät generell für diese Anwendung nutzbar ist, welche der vier Methoden zum Erreichen der Firmenziele optimal ist und ob die Methoden genauere Ergebnisse liefern können, als der ursprüngliche Algorithmus.




