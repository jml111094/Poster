\textbf{\Large Einleitung}\\

Die Spiroergometrie (aus lat. \textsl{spirare}: atmen, griech. \textsl{ergo}: Arbeit) ist eine sportmedizinisch-technische Anwendung, bei der respiratorische Daten während inkrementierter körperlicher Belastung erfasst und anschließend analysiert werden. Anhand dieser Daten können z.B. individuelle Trainingsbereiche bestimmt werden. Der Hamburger Medizintechnik-Hersteller cardioscan GmbH hat 2017 ein Spiroergometer entwickelt, welches in Verbindung mit einer Software zur spiroergometrischen Bestimmung der ventilatorischen Schwellen VT1 und VT2 genutzt werden soll. Die Software verwendet momentan einen VT2-Algorithmus , der recht sensitiv für Fehler ist und deshalb durch einen neuen ersetzt werden muss.\\
Es wurden nach der Literaturrecherche insgesamt vier wissenschaftlich empfohlene Methoden ausgewählt~\cite{Westhoff.2012}, die in Folge an eine Spiroergometrie mit dem neuen Gerät des Unternehmens zur Schwellenbestimmung verwendet wurden. Zu überprüfen war, ob das Gerät generell für diese Anwendung nutzbar ist, welche der vier Methoden zum Erreichen der Firmenziele optimal ist und ob die Methoden genauere Ergebnisse liefern können, als der ursprüngliche Algorithmus.\\





