\begin{center} \textbf{\Large Methode} \end{center}

In einem firmeninternen Projekt wurden mit 28 unterschiedlichen Probanden nach einem gleichen festgelegten Prozedere spiroergometrische Tests auf einem Fahrradergometer durchgeführt, bei der die Herzfrequenz zusätzlich über einen Pulsgurt erfasst wurde. Die Sensor-Rohdaten des Spiroergometers, die Herzfrequenz sowie die Leistungswerte des Ergometers wurden durch ein MATLAB-Programm weiterverarbeitet und grafisch in Form von eigens erstellten "`6-Felder-Grafiken"' visualisiert. In diesen sind die Plots für alle vier Methoden enthalten. Die Grafiken wurden subjektiv von zwei Ratern sowie mathematisch durch einen Algorithmus analysiert. Anschließend wurden die Ergebnisse der unterschiedlichen Methoden statistisch miteinander verglichen und Differenzen bzw. Abweichungen bei den identifizierten Schwellen untersucht. Die Ergebnisse wurden außerdem mit den Erkenntnissen der HUNT 3 Studie aus dem Jahre 2014 verglichen, um zu diskutieren, ob die individuell gemessenen Werte realistisch waren.
