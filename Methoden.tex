\textbf{\Large Methode}\\

In einem firmeninternen Projekt wurden mit 28 unterschiedlichen Probanden spiroergometrische Tests auf einem Fahrradergometer absolviert, bei der die Herzfrequenz zusätzlich über einen Pulsgurt erfasst wurde. Die Durchführung folgte einem gleichbleibenden festgelegten Prozedere mit jedoch individuellen Belastungsprotokollen gemäß des Trainingszustands einer Person.\\
Die Sensor-Rohdaten des Spiroergometers, die Herzfrequenz sowie die Leistungswerte des Ergometers wurden durch die Software gespeichert, durch ein MATLAB-Programm weiterverarbeitet und grafisch in Form von "`6-Felder-Grafiken"' visualisiert. In diesen sind die Plots für alle vier Methoden enthalten. Als VT2-Referenzmethode diente außerdem der ursprüngliche Algorithmus. Die Grafiken wurden subjektiv von zwei Ratern sowie mathematisch durch einen Algorithmus analysiert. Anschließend wurden die Ergebnisse der unterschiedlichen Methoden statistisch miteinander verglichen und Differenzen bzw. Übereinstimmungen bei den identifizierten Schwellen untersucht. Die Ergebnisse wurden außerdem mit den Erkenntnissen der HUNT 3 Studie aus dem Jahre 2014 verglichen, um zu diskutieren, ob die gemessenen Werte realistisch waren.